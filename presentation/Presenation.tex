% Copyright 2004 by Till Tantau <tantau@users.sourceforge.net>.
%
% In principle, this file can be redistributed and/or modified under
% the terms of the GNU Public License, version 2.
%
% However, this file is supposed to be a template to be modified
% for your own needs. For this reason, if you use this file as a
% template and not specifically distribute it as part of a another
% package/program, I grant the extra permission to freely copy and
% modify this file as you see fit and even to delete this copyright
% notice. 

\documentclass{beamer}

\usepackage{fontspec} % Need this to use Optima
\setsansfont{Optima} % Set main font here

% There are many different themes available for Beamer. A comprehensive
% list with examples is given here:
% http://deic.uab.es/~iblanes/beamer_gallery/index_by_theme.html
% You can uncomment the themes below if you would like to use a different
% one:
%\usetheme{AnnArbor}
%\usetheme{Antibes}
%\usetheme{Bergen}
%\usetheme{Berkeley}
%\usetheme{Berlin}
%\usetheme{Boadilla}
%\usetheme{boxes}
%\usetheme{CambridgeUS}
%\usetheme{Copenhagen}
%\usetheme{Darmstadt}
%\usetheme{default}
%\usetheme{Frankfurt}
%\usetheme{Goettingen}
%\usetheme{Hannover}
%\usetheme{Ilmenau}
%\usetheme{JuanLesPins}
%\usetheme{Luebeck}
\usetheme{Madrid}
%\usetheme{Malmoe}
%\usetheme{Marburg}
%\usetheme{Montpellier}
%\usetheme{PaloAlto}
%\usetheme{Pittsburgh}
%\usetheme{Rochester}
%\usetheme{Singapore}
%\usetheme{Szeged}
%\usetheme{Warsaw}

\title{Computational Analysis of Classical Texts}

% A subtitle is optional and this may be deleted
%\subtitle{Optional Subtitle}

\author{Marc E. Canby}
% - Give the names in the same order as the appear in the paper.
% - Use the \inst{?} command only if the authors have different
%   affiliation.

\institute[] % (optional, but mostly needed)
{
	\inst{}%
	Lati 318 $-$ Cicero: \textit{De Re Publica}\\
	Rice University

%	\inst{2}%
%	Department of Theoretical Philosophy\\
%	University of Elsewhere
}
% - Use the \inst command only if there are several affiliations.
% - Keep it simple, no one is interested in your street address.

\date{}
% - Either use conference name or its abbreviation.
% - Not really informative to the audience, more for people (including
%   yourself) who are reading the slides online

%\subject{Theoretical Computer Science}
% This is only inserted into the PDF information catalog. Can be left
% out. 

% If you have a file called "university-logo-filename.xxx", where xxx
% is a graphic format that can be processed by latex or pdflatex,
% resp., then you can add a logo as follows:

% \pgfdeclareimage[height=0.5cm]{university-logo}{university-logo-filename}
% \logo{\pgfuseimage{university-logo}}

% Delete this, if you do not want the table of contents to pop up at
% the beginning of each subsection:
\AtBeginSection[]
{
\begin{frame}<beamer>{Outline}
\tableofcontents
[
currentsection,
%currentsubsection,
%subsectionstyle=show
subsectionstyle=show/show/shaded
]
\end{frame}
}

% Change numbers in table of contents
\setbeamertemplate{section in toc}{\inserttocsectionnumber.~\inserttocsection}

% make symbols bold
\newcommand{\bld}[1]{\boldsymbol{#1}}

% for long division
\usepackage{array}
\usepackage{polynom}

% algorithms
\usepackage[]{algorithm2e}

\usepackage{tabto} % tabbing

% Let's get started
\begin{document}

\begin{frame}
\titlepage
\end{frame}

\begin{frame}{Outline}
\tableofcontents
\end{frame}

\section{Getting and Cleaning the Data}

\begin{frame}{Getting and Cleaning the Data}


\begin{itemize}
	\setlength\itemsep{1em}
	\item Text obtained from \textit{The Latin Library} at the sentence level:
	\begin{itemize}
		\item {\ttfamily ['nempe',
		'ab',
		'iis',
		'qui',
		'haec',
		'disciplinis',
		'informata',
		'alia',
		'moribus',
		'confirmarunt',
		',',
		'sanxerunt',
		'autem',
		'alia',
		'legibus',
		'.']}
	\end{itemize}
\item Cleaned up messy elements of data:
\begin{itemize}
	\setlength\itemsep{0.5em}
	\item Line numbers: {\ttfamily [1,2,...,71]}
	\item Angle brackets: {\ttfamily ['\&',
		'lt',
		';',
		'im\&gt',
		';',
		'petu',
		'liberavissent',
		',',
		'nec',...]}
	\begin{itemize}
		\item {\ttfamily \&lt;} should be {\ttfamily <} \qquad {\ttfamily \&gt;} should be {\ttfamily >}
	\end{itemize}
\item Hyphens encoded as {\ttfamily \&\#}
\item English words: {\ttfamily ['Cicero', 'The', 'Latin', 'Library', 'The', 'Classics', 'Page']}
\end{itemize}
\end{itemize}

\end{frame}




\section{Exploratory Text Analysis}

\begin{frame}{Exploratory Text Analysis: Tokenization and POS Tagging}
\begin{itemize}
	\item Map each word to its base form (\textit{lemma} or \textit{token}) and its POS
	\item Often ignore \textit{stop words} ('et', 'sum', etc.) $-$ highlighted in red
\end{itemize}
\begin{center}
	{\ttfamily ['nempe',
		'ab',
		'iis',
		'qui',
		'haec',
		'disciplinis',
		'informata',
		'alia',
		'moribus',
		'confirmarunt',
		',',
		'sanxerunt',
		'autem',
		'alia',
		'legibus',
		'.']}
	\vspace{12pt}
	
	{\ttfamily[('nempe', 'adverb'),
		{\color{red} ('ab', 'preposition')},
		{\color{red}('is', 'pronoun')},
		{\color{red}('qui', 'pronoun')},
		{\color{red}('hic', 'pronoun')},
		('disciplina', 'noun'),
		('informo', 'noun'),
		('alius2', 'adjective'),
		('mos', 'noun'),
		('confirmo', 'verb'),
		('sancio', 'verb'),
		{\color{red}('autem', 'conjunction')},
		('alius2', 'adjective'),
		('lex', 'noun')]}
	
\end{center}
\end{frame}

\begin{frame}{Exploratory Text Analysis}
\begin{itemize}
	\item Number of characters: 109,777 \tab (\textit{average:} 136,893)
	\item Number of words: 20,067 \tab (\textit{average:} 24,924)
	\item Number of sentences: 820 \tab (\textit{average:} 1,059)
\end{itemize}



\begin{figure}
	\centering
	\begin{minipage}{.5\textwidth}
		\centering
		\includegraphics[scale=0.19]{fig1.png}
	\end{minipage}%
	\begin{minipage}{.5\textwidth}
		\centering
		\includegraphics[scale=0.19]{fig2.png}
	\end{minipage}
\end{figure}
\end{frame}
\section{Keyword Extraction: Frequency Count and TextRank}

\begin{frame}{Keyword Extraction: Frequency Count}
\begin{itemize}
		\setlength\itemsep{1em}
	\item Goal: Determine a set of keywords that summarizes the text
	\item Naive approach: Take words in text with highest frequency
	\begin{center}
	\begin{tabular}{| c | c || c | c |}
		\hline
		\textbf{Word} & \textbf{Frequency} & \textbf{Word} & \textbf{Frequency} \\ \hline
		res & 191 & magnus & 74 \\ \hline
		publicus & 163 & civitas & 72 \\ \hline
		Scipio & 110 & bonus & 70  \\ \hline
		populus & 103 & Laelius & 66  \\ \hline
		homo & 94 & rex & 51  \\
		\hline
\end{tabular}\\
\footnotesize{Note: Restricted to nouns and adjectives only}
\end{center}
	
	\item Problem: Does not account for structure of text and relationships between words
\end{itemize}
\end{frame}

\begin{frame}{Keyword Extraction: TextRank}
TextRank is a popular keyword extraction algorithm that determines the importance of a keyword based on its relationship to other keywords.
\begin{itemize}

	\item Step 1: Build graph representing text
	\item Step 2: Run keyword extraction algorithm on graph
	\item Step 3: Merge neighboring keywords and recompute scores
\end{itemize}
\end{frame}


\begin{frame}{TextRank: Step 1 (Build graph representing text)}

\end{frame}

\section{Predicting Missing Text: LSTM Neural Networks}

% t bound in previous slide
% simple example
% encoding
%  - multiplication and expansion
%  - machine
% decoding
%  - whole syndrome thing
% why this is the most useful code
% alternate view - n points
% applications

%\begin{itemize}
%\item {
%First item.
%\pause % The slide will pause after showing the first item
%}
%\item {   
%Second item.
%}
%% You can also specify when the content should appear
%% by using <n->:
%\item<3-> {
%Third item.
%}
%\item<4-> {
%Fourth item.
%}
%% or you can use the \uncover command to reveal general
%% content (not just \items):
%\item<5-> {
%Fifth item. \uncover<6->{Extra text in the fifth item.}
%}
%\end{itemize}



%\begin{frame}{Blocks}
%\begin{block}{Block Title}
%You can also highlight sections of your presentation in a block, with it's own title
%\end{block}
%\begin{theorem}
%There are separate environments for theorems, examples, definitions and proofs.
%\end{theorem}
%\begin{example}
%Here is an example of an example block.
%\end{example}
%\end{frame}

% Placing a * after \section means it will not show in the
% outline or table of contents.
%\section*{Summary}

%\begin{frame}{Summary}
%\begin{itemize}
%\item
%The \alert{first main message} of your talk in one or two lines.
%\item
%The \alert{second main message} of your talk in one or two lines.
%\item
%Perhaps a \alert{third message}, but not more than that.
%\end{itemize}
%
%\begin{itemize}
%\item
%Outlook
%\begin{itemize}
%\item
%Something you haven't solved.
%\item
%Something else you haven't solved.
%\end{itemize}
%\end{itemize}
%\end{frame}



% All of the following is optional and typically not needed. 
%\appendix
%\section<presentation>*{\appendixname}
%\subsection<presentation>*{For Further Reading}
%
%\begin{frame}[allowframebreaks]
%\frametitle<presentation>{For Further Reading}
%
%\begin{thebibliography}{10}
%
%\beamertemplatebookbibitems
%% Start with overview books.
%
%\bibitem{Author1990}
%A.~Author.
%\newblock {\em Handbook of Everything}.
%\newblock Some Press, 1990.
%
%
%\beamertemplatearticlebibitems
%% Followed by interesting articles. Keep the list short. 
%
%\bibitem{Someone2000}
%S.~Someone.
%\newblock On this and that.
%\newblock {\em Journal of This and That}, 2(1):50--100,
%2000.
%\end{thebibliography}
%\end{frame}

\end{document}


